\documentclass{beamer}
\usepackage{beamerthemeAnnArbor}
\usepackage{amsfonts}
\usepackage{amssymb}
\usepackage{ulem}              

\def\da{\delta(\alpha)}

\setbeamercolor{rosu}{fg=red}
\setbeamerfont{fontMy}{size=\small}

\theoremstyle{plain}
\newtheorem{de}{Definition}[section]
\newtheorem{lm}[de]{Lemma}
\newtheorem{re}[de]{Observation}
\newtheorem{te}[de]{Theorem}
\newtheorem{ex}[de]{Example}
\newtheorem{Prop}{Propozition}

\def\be{\begin{equation}}
\def\ee{\end{equation}}
\def\l{\langle}
\def\r{\rangle}
\def\p{\partial}
\def\R{I\!\!R}
\def\12{\frac{1}{2}}
\def\sr1{1, \dots, r}
\def\a{\alpha}
\def\lmi{\lambda_{min}}
\def\lma{\lambda_{max}}
\def\ah{\hat{A}}
\def\bh{\hat{b}}
\def\xh{\hat{x}}
\def\yh{\hat{y}}
\def\ss{\sigma^*}
\def\s{\sigma}
\def\ab{\bar{A}}
\def\mb{\bar{M}}
\def\rb{\bar{R}}
\def\G{\Gamma}

% \useoutertheme[footline=empty]{shadow}
% \usecolortheme{seagull}  %outer color
% \usecolortheme{default}   %inner color
% \setbeamerfont{title}{shape=\itshape,family=\rmfamily}
% \setbeamercolor{title}{fg=red!80!black}
% \setbeamercolor{title}{fg=red!80!black,bg=red!20!white}

\usetheme{AnnArbor}
\usecolortheme{beaver}
\setbeamercolor{titlelike}{parent=structure,bg=white}

%culori activare/dezactivare
\def\colorize<#1>{                     
\temporal<#1>{\color{black}}{\color{black}}{\color{black!50}}}
%\usefonttheme{serif} % fontul serif is the best ...
% Titlu
\title
% [Cuprins]
{\textbf{{\c Siruri}}}
\subtitle{}

%Autor
\author{\textbf{T\u anase Ramona Elena }}

%Institutia
\institute{Universitatea Ovidius Constanta\\
Facultatea de Matematic\u a \c si Informatic\u a\\
Specializarea:Matematic\u a - Informatic\u a}
%$^{\ast\ast}$Friedrich-Alexander Universit\"{a}t Erlangen-N\"{u}rnberg, Germany
%}           

%Data & informatii despre prezentare         
\date{
\textbf{
Iulie, 2021}\\
\bigskip
}

\begin{document}

\frame{\titlepage}
\frame
{
\frametitle{Cuprins}
\begin{itemize}
	\item[1.] \c Siruri
	\item[2.] \c Siruri convergente de numere reale 
	\item[3.] \c Siruri m\u arginite
	\item[4.] \c Siruri recurente \c si asimtote oblice
	\item[5.] Bibliografie
\end{itemize}
}
%%%%%%%%%%%%%%%%%%%%%%%%%%%%%%%%%%%%%%%%%%%%%%%%%%%%%%%%%%%%%%%%%%%%%%%%%%%%%%%%%%%%%%%%%%%%%%%%%%
\frame
{
\frametitle{\c Siruri}

\begin{definition}
Fie X o mulțime. O funcție \(f:\mathbb{N} \to X\) se numește șir de elemente din mulțimea X, sau sub o altă formulare: se numește șir de elemente din mulțimea X o funcție \(f:\mathbb{N} \to X\) . În mod uzual, se notează \(f_{1} = x_{1} \in X, f_{2} = x_{2} \in X,......, f_{n} = x_{n} \in X,....\)
\end{definition}

}
%%%%%%%%%%%%%%%%%%%%%%%%%%%%%%%%%%%%%%%%%%%%%%%%%%%%%%%%%%%%%%%%%%%%%%%%%%%%%%%%%%%%%%%%%%%%%%%%%%
\frame
{
\frametitle{\c Siruri convergente de numere reale }
Teorie

\begin{definition}

Un șir \((x_{n})_{n \in \mathbb{N}} \subset \mathbb{R}\) , se numește convergent dacă există \(x \in \mathbb{R}\) astfel încât:
	\(\forall _{\varepsilon } > 0, \in n_{\varepsilon } \in \mathbb{N}\) astfel încât este satisfacută inegalitatea:  \(\left | x_{n}- x \right | \leq \varepsilon\) . 
	\end{definition}
}
%%%%%%%%%%%%%%%%%%%%%%%%%%%%%%%%%%%%%%%%%%%%%%%%%%%%%%%%%%%%%%%%%%%%%%%%%%%%%%%%%%%%%%%%%%%%%%%%%%
\frame
{
\frametitle{\c Siruri convergente de numere reale}
Propoziție

Unicitatea limitei unui șir de numere reale

Fie \((x_{n})_{n \in \mathbb{N}} \subset \mathbb{R}\) . 

Dacă
\begin{displaymath}
  \left\{\begin{matrix}
x_{n} \to  x\\ 
x_{n} \to y
\end{matrix}\right.
\end{displaymath}

atunci \(x = y\).

}
%%%%%%%%%%%%%%%%%%%%%%%%%%%%%%%%%%%%%%%%%%%%%%%%%%%%%%%%%%%%%%%%%%%%%%%%%%%%%%%%%%%%%%%%%%%%%%%%%%
\frame
{
\frametitle{\c Siruri convergente de numere reale}
Demonstrație

Să presupunem, prin absurd, că \(x \neq  y\) . Cum suntem pe \(\mathbb{R}\) înseamnă că avem una din situațiile \(x < y\) sau \(y < x\). Pentru a face o alegere, fie \(x < y\) atunci \(y – x > 0\) și din definiție pentru \(\varepsilon = \frac{y- x}{2}  > 0\) rezultă că, 
\(\exists  n_{1} \in \mathbb{N}\) astfel încât \(\left | x_{n} - x  \right | < \frac{y - x }{2} , \forall n \geq n_{1} \)
și 
\(\exists  n_{2} \in \mathbb{N}\) astfel încât \(\left | x_{n} - y  \right | < \frac{y - x }{2} , \forall n \geq n_{2}\) 
Fie \(n = max (n _{1}, n_{2}) \geq n_{1}, n_{2}.\) Atunci: 
\(\left | x_{n} - x \right | < \frac{y-x}{2}\) și \(\left | x_{n} - y  \right | <  \frac{y-x}{2}\)
de unde 
\begin{displaymath}
  y-x = \left | y-x \right | = \left | (y-x_{n})+ (x_{n} -x) \right |\leq \left | y-x_{n} \right | + \left | x_{n} - x \right | < \frac{y-x}{2} + \frac{y-x}{2} = y-x
\end{displaymath}

	Așadar, \(y-x < y-x\) , contradicție!
}
%%%%%%%%%%%%%%%%%%%%%%%%%%%%%%%%%%%%%%%%%%%%%%%%%%%%%%%%%%%%%%%%%%%%%%%%%%%%%%%%%%%%%%%%%%%%%%%%%%
\frame
{
\frametitle{\c Siruri convergente de numere reale}
Exerciții

Calculați 

\begin{displaymath}
  \lim_{n\to\infty }\left ( \frac{1}{\sqrt{n^{4}+1}}+ \frac{2}{\sqrt{n^{4}+2} } +\frac{3}{\sqrt{n^{4}+3}}+........+\frac{n}{\sqrt{n^{4}+n}}  \right )
\end{displaymath}
Rezolvare

Notăm \(x_{n}= \frac{1}{\sqrt{n^{4}+1}} + \frac{2}{\sqrt{n^{4}+2}}+\frac{3}{\sqrt{n^{4}+3}}+........+\frac{n}{\sqrt{n^{4}+n}}\).
\\ Adică \(x_{n}= \sum_{k=1}^{n}\frac{k}{n^{4}+k}\).

În continuare procedăm astfel. De numărător nu ne atingem. Vom lucra cu numitorul, ideea fiind de a se avea același numitor peste tot. 
\\ Avem \(1\leq k\leq n\) de unde \(n^{4}+1 \leq n^{4}+k \leq n^{4}+n\) de unde\\  \(\sqrt{n^{4}+1}\leq \sqrt{n^{4}+k}\leq \sqrt{n^{4}+1}\) de unde \\ \(\frac{1}{\sqrt{n^{4}+1}}\geq \frac{1}{\sqrt{n^{4}+k}}\geq \frac{1}{\sqrt{n^{4}+n}}\).

}
%%%%%%%%%%%%%%%%%%%%%%%%%%%%%%%%%%%%%%%%%%%%%%%%%%%%%%%%%%%%%%%%%%%%%%%%%%%%%%%%%%%%%%%%%%%%%%%%%%
\frame
{
\frametitle{\c Siruri convergente de numere reale}


Acum înmulțind cu k obținem 
\begin{displaymath}
\frac{k}{\sqrt{n^{4}+1}}\geq \frac{k}{\sqrt{n^{4}+k}}\geq \frac{k}{\sqrt{n^{4}+n}} \label{eq:1.1} \tag{1.1}
\end{displaymath}
\\ În continuare în relația \ref{eq:1.1} dam lui k valorile \(1,2,.....,n\). 
\\Pentru \( k = 1\) rezultă
\(\frac{1}{\sqrt{n^{4}+1}}\geq \frac{1}{\sqrt{n^{4}+1}}\geq \frac{1}{\sqrt{n^{4}+n}}\)
\\ Pentru \( k = 2\) rezultă
 \(\frac{2}{\sqrt{n^{4}+1}}\geq \frac{2}{\sqrt{n^{4}+2}}\geq \frac{2}{\sqrt{n^{4}+n}}\)
\\ Adunând inegalitățile de mai sus obținem 

\begin{displaymath}
 \frac{1}{\sqrt{n^{4}+1}}+ \frac{2}{\sqrt{n^{4}+1}}+......+ \frac{n}{\sqrt{n^{4}+1}} \geq 
\end{displaymath}
 
\begin{displaymath}
 \geq \frac{1}{\sqrt{n^{4}+1}}+ \frac{2}{\sqrt{n^{4}+2}}+......+ \frac{n}{\sqrt{n^{4}+n}}\geq
\end{displaymath}
 
\begin{displaymath}
 \geq \frac{1}{\sqrt{n^{4}+n}}+ \frac{2}{\sqrt{n^{4}+n}}+......+ \frac{n}{\sqrt{n^{4}+n}}
\end{displaymath}

}
%%%%%%%%%%%%%%%%%%%%%%%%%%%%%%%%%%%%%%%%%%%%%%%%%%%%%%%%%%%%%%%%%%%%%%%%%%%%%%%%%%%%%%%%%%%%%%%%%%
\frame
{
\frametitle{\c Siruri convergente de numere reale}
Sau \(\frac{1+2+....+n}{\sqrt{n^{4}+1}}\geq x_{n}\geq \frac{1+2+.....+n}{\sqrt{n^{4}+n}}\)
 \\ Dar știm că \(1+2+...+n = \frac{n(n+1)}{2}\), 
 deci vom obține \(\frac{n(n+1)}{2\sqrt{n^{4}+1}}\geq x_{n}\geq \frac{n(n+1)}{2\sqrt{n^{4}+n}}\). (2)
\\ Acum 
\begin{displaymath}
 \lim_{n \to \infty }\frac{n(n+1)}{2\sqrt{n^{4}+1}}=\frac{1}{2} și \lim_{n \to \infty }\frac{n(n+1)}{2\sqrt{n^{4}+n}}=\frac{1}{2}. (3)
\end{displaymath}
\\ Vom da la ambele factor comun forțat. Din (2) și (3) și teorema cleștelui rezultă că
\begin{displaymath}
 \lim_{n \to \infty }x_{n}=\frac{1}{2}
\end{displaymath}
}
%%%%%%%%%%%%%%%%%%%%%%%%%%%%%%%%%%%%%%%%%%%%%%%%%%%%%%%%%%%%%%%%%%%%%%%%%%%%%%%%%%%%%%%%%%%%%%%%%%
\frame
{
\frametitle{\c Siruri m\u arginite}
Teorie 
\\ Definiție 
\\ Fie \((x_{n})_{n\in \mathbb{N}}\) un șir de numere reale. 
\\ Șirul \((x_{n})_{n\in \mathbb{N}}\) se numește mărginit dacă și numai dacă \(\exists  a, b \in \mathbb{R}\), a< b astfel încât \(\forall n\in \mathbb{N}\) este satisfacută inegalitatea \(x_{n}\in \left [ a,b \right ]\), sau echivalent \(\exists M> 0\) astefle încât \(\forall  n\in \mathbb{N}\) este satisfacută inegalitatea \(\left | x_{n} \right |\leq M\).
\\ Definiție 
\\ Fie \((x_{n})_{n\in \mathbb{N}}\) un șir de numere reale. Spunem că \(\lim_{n \to \infty }x_{n}=\infty\) dacă, \(\forall \varepsilon > 0,\exists n_{\varepsilon }\in \mathbb{N}\) astfel încât pentru \(\forall n\geq n_{\varepsilon }\) este satisfacută inegalitatea \(x_{n}> \varepsilon\). 
\\Sau \(\forall \varepsilon > 0,\exists n_{\varepsilon }\in \mathbb{N}\) astfel încât \(x_{n}> \varepsilon ,\forall n\geq n_{\varepsilon }\). 

}
%%%%%%%%%%%%%%%%%%%%%%%%%%%%%%%%%%%%%%%%%%%%%%%%%%%%%%%%%%%%%%%%%%%%%%%%%%%%%%%%%%%%%%%%%%%%%%%%%%
\frame
{
\frametitle{\c Siruri m\u arginite}
Exerciții
\\ Calculați
\\ Fie \(\alpha > 0\) să se calculeze 
\begin{displaymath}
 \lim_{n \to \infty }\frac{1^{\alpha }+2^{\alpha }+....+n^{\alpha }}{n^{\alpha +1}}
\end{displaymath}
\\ Demonstrație 
\\ Fie \(x_{n}=1^{\alpha }+2^{\alpha }+....+n^{\alpha },a_{n}= n^{\alpha }\).  Deoarece \(\alpha > 0 , \alpha \uparrow \infty\). 
\\ Din lema Stolz-Cesaro, cazul \(\left [ \frac{1}{\infty } \right ]\), 
\begin{displaymath}
 \lim_{n \to \infty }\frac{1^{\alpha }+2^{\alpha }+....+n^{\alpha }}{n^{\alpha +1}}=\lim_{n \to \infty }\frac{x_{n}}{\alpha _{n}}=\lim_{n \to \infty } \frac{x_{n+1}-x_{n}}{\alpha _{n+1}-_{n}}=
 
\end{displaymath}
\begin{displaymath}
 =\lim_{n \to \infty } \frac{\left ( n+1 \right )^{\alpha }}{\left ( n+1 \right )^{\alpha+1} -n^{\alpha +1}}
\end{displaymath}
}
%%%%%%%%%%%%%%%%%%%%%%%%%%%%%%%%%%%%%%%%%%%%%%%%%%%%%%%%%%%%%%%%%%%%%%%%%%%%%%%%%%%%%%%%%%%%%%%%%%
\frame
{
\frametitle{\c Siruri m\u arginite}
\begin{displaymath}
 \lim_{n \to \infty }\frac{\left ( n+1 \right )^{\alpha +1}-n^{\alpha +1}}{\left ( n+1 \right )^{\alpha }}
\end{displaymath}
\\ Dăm factor comun forțat la numărător pe \(n^{\alpha +1}\). 
\\ Avem 
\begin{displaymath}
 \lim_{n \to \infty }\frac{\left ( n+1 \right )^{\alpha +1}-n^{\alpha +1}}{\left ( n+1 \right )^{\alpha }} = \lim_{n \to \infty }\frac{n^{\alpha +1\left [ \frac{\left ( n+1 \right )^{\alpha +1}}{n^{\alpha +1}} -1\right ]}}{\left ( n+1 \right )^{\alpha }} = 
\end{displaymath}
\begin{displaymath}
 = \lim_{n \to \infty }\frac{n^{\alpha }}{\left ( n+1 \right )^{\alpha }}\cdot n\left [ \left ( \frac{n+1}{n} \right )^{\alpha +1}-1 \right ] = 
\end{displaymath}
\begin{displaymath}
 =\lim_{n \to \infty }\frac{n^{\alpha }}{\left ( n+1 \right )^{\alpha }}\cdot \lim_{n \to \infty }n\left [ \left ( \frac{n+1}{n} \right )^{\alpha +1} -1\right ]=
\end{displaymath}
\begin{displaymath}
 =\lim_{n \to \infty }n\left [ \left ( 1+\frac{1}{n} \right )^{\alpha +1}-1 \right ]= 
\end{displaymath}
}
%%%%%%%%%%%%%%%%%%%%%%%%%%%%%%%%%%%%%%%%%%%%%%%%%%%%%%%%%%%%%%%%%%%%%%%%%%%%%%%%%%%%%%%%%%%%%%%%%%
\frame
{
\frametitle{\c Siruri m\u arginite}
\begin{displaymath}
 = \lim_{n \to \infty }\frac{\left ( 1+\frac{1}{n} \right )^{\alpha +1}-1}{\frac{1}{n}}=
\end{displaymath}
\begin{displaymath}
 =\lim_{n \to \infty }\frac{\left ( 1+n \right )^{\alpha +1}-1}{n}= \alpha +1=
\end{displaymath}
\\ Am folosit limita fundamentală 
\begin{displaymath}
 \lim_{x \to \infty }\frac{\left ( 1+x \right )^{\gamma }-1}{x} = \gamma ,\gamma \in \mathbb{R}
\end{displaymath}
\\ Întorcându-ne la problemă, obținem:
\begin{displaymath}
 \lim_{n \to \infty }\frac{1^{\alpha }+2^{\alpha }+....+n^{\alpha }}{n^{\alpha +1}} = \frac{1}{\alpha +1}
\end{displaymath}
}
%%%%%%%%%%%%%%%%%%%%%%%%%%%%%%%%%%%%%%%%%%%%%%%%%%%%%%%%%%%%%%%%%%%%%%%%%%%%%%%%%%%%%%%%%%%%%%%%%%
\frame
{
\frametitle{\c Siruri recurente \c si asimtote oblice}
Teorie
\\ Teoremă
\\ Fie \(a\in \mathbb{R}\) și \(f: \left ( \alpha ,\infty  \right )n \to \mathbb{R}\) o funcție continuă cu proprietatea că \(f_{(x)}> x, \forall x > a\). Definim șirul de numere reale \(\left ( x_{n} \right )_{n\geq 1}\) prin condiția inițială \(x_{1}> \alpha\) și relația de recurență \(x_{n+1} = f_{\left ( x_{n} \right )} pentru orice  n\geq 1\).
\\ Atunci 
\begin{displaymath}
 \lim_{x \to \infty }x_{n} = \infty
\end{displaymath}
\\ Dacă există \(b_{0}\in \mathbb{R}\) astfel încât \(y = x + b_{0}\) este asimtotă oblică la graficul funcției f, atunci
\(\lim_{x \to \infty }\frac{x_{n}}{n}=b_{0}\)
}
%%%%%%%%%%%%%%%%%%%%%%%%%%%%%%%%%%%%%%%%%%%%%%%%%%%%%%%%%%%%%%%%%%%%%%%%%%%%%%%%%%%%%%%%%%%%%%%%%%
\frame
{
\frametitle{\c Siruri recurente \c si asimtote oblice}
Dacă există \(b_{0}, b_{1}\in \mathbb{R}, b_{0 }\neq 0\) astfel încât 
\(\lim_{n \to \infty }x\left ( f\left ( x \right )-x-b_{0} \right )= b_{1},
\lim_{n \to \infty } \frac{n}{\ln n}\left ( \frac{x_{n}}{n} -b_{0}\right )=\frac{b_{1}}{b_{0}}\)

}
%%%%%%%%%%%%%%%%%%%%%%%%%%%%%%%%%%%%%%%%%%%%%%%%%%%%%%%%%%%%%%%%%%%%%%%%%%%%%%%%%%%%%%%%%%%%%%%%%%
\frame
{
\frametitle{\c Siruri recurente \c si asimtote oblice}
Exerciții
\\ Calculați
\begin{displaymath}
 \lim_{n \to \infty }\frac{\sum_{k=1}^{n}k\left ( \sqrt[n]{n+k} -1\right )}{n\ln n } = \frac{1}{2}
\end{displaymath}
\\ Demonstrație 
\\ Să notăm \(x_{n} = \frac{1}{n}\sum_{k=1}^{n} k \ln \left ( n+k \right ), n \geq 1\). 
\begin{displaymath}
 \sum_{k=1}^{n}k\left ( \sqrt[n]{n+k}-1 \right )\sim x_{n}. \label{eq:2.1} \tag{2.1}
\end{displaymath}
}
%%%%%%%%%%%%%%%%%%%%%%%%%%%%%%%%%%%%%%%%%%%%%%%%%%%%%%%%%%%%%%%%%%%%%%%%%%%%%%%%%%%%%%%%%%%%%%%%%%
\frame
{
\frametitle{\c Siruri recurente \c si asimtote oblice}
Fie \(n\geq 2\). 
\\ Avem \(\ln \left ( n+1 \right )\leq \ln \left ( n+k \right )\leq \ln \left ( n+n \right ), \forall 1\leq k\leq n\), de unde 
\begin{displaymath}
 \sum_{k=1}^{n} k \ln \left ( n+1 \right )\leq \sum_{k=1}^{n}k \ln \left ( n+k \right )\leq \sum_{k=1}^{n} k \ln \left ( n+n \right ), \frac{\ln \left ( n+1 \right )}{\ln n }
\end{displaymath}
\begin{displaymath}
 \frac{\sum_{k=1}^{n}k}{n^{2}}\leq \frac{x_{n}}{n\ln n}\leq \frac{\ln \left ( n+n \right )}{\ln n }
\end{displaymath}
\begin{displaymath}
 \frac{\sum_{k=1}^{n}k}{n^{2}},
\end{displaymath}
 sau încă
}
%%%%%%%%%%%%%%%%%%%%%%%%%%%%%%%%%%%%%%%%%%%%%%%%%%%%%%%%%%%%%%%%%%%%%%%%%%%%%%%%%%%%%%%%%%%%%%%%%%
\frame
{
\frametitle{\c Siruri recurente \c si asimtote oblice}
\begin{displaymath}
 \frac{\ln \left ( n+1 \right )}{\ln n} \cdot \frac{n+1}{2n}\leq \frac{x_{n}}{n\ln n }\leq \frac{\ln \left ( n+n \right )}{\ln n }\cdot \frac{n+1}{2n} \label{eq:2.2} \tag{2.2}
\end{displaymath}
\\ Din relația \ref{eq:2.2} și teorema cleștelui rezultă că \(\lim_{n \to \infty }\frac{x_{n}}{n\ln n } = \frac{1}{2}\). Astfel spus 

\begin{displaymath}
x_{n\sim }\frac{n\ln n }{2} \label{eq:2.3} \tag{2.3}
\end{displaymath}

Din relațiile \ref{eq:2.1} și \ref{eq:2.3} rezultă că \(\sum_{k=1}^{n}k\left ( \sqrt[n]{n+k}-1 \right )\sim \frac{n\ln n }{2}\), adică egalitatea din enunț. 

}
%%%%%%%%%%%%%%%%%%%%%%%%%%%%%%%%%%%%%%%%%%%%%%%%%%%%%%%%%%%%%%%%%%%%%%%%%%%%%%%%%%%%%%%%%%%%%%%%%%
\frame
{
\frametitle{Bibliografie}
\begin{itemize}
\item[(1)] Popa,Dumitru, Curs Matematică didactică – Analiză – Capitole speciale de analiză matematică pentru pregătirea profesorilor , 2020-2021
\end{itemize}
}
\frame
{\frametitle{Multumiri}
\begin{center}
{\Large V\u a mul\c tumesc pentru aten\c tie !}	
\end{center}
}
%%%%%%%%%%%%%%%%%%%%%%%%%%%%%%%%%%%%%%%%%%%%%%%%%%%%%%%%%%%%%%%%%%%%%%%%%%%%%%%%%%%%%%%%%%%%%%%%%%
\end{document}
\documentclass{beamer}
\usepackage{beamerthemeAnnArbor}
\usepackage{amsfonts}
\usepackage{amssymb}
\usepackage{ulem}              

\def\da{\delta(\alpha)}

\setbeamercolor{rosu}{fg=red}
\setbeamerfont{fontMy}{size=\small}

\theoremstyle{plain}
\newtheorem{de}{Definition}[section]
\newtheorem{lm}[de]{Lemma}
\newtheorem{re}[de]{Observation}
\newtheorem{te}[de]{Theorem}
\newtheorem{ex}[de]{Example}

\def\be{\begin{equation}}
\def\ee{\end{equation}}
\def\l{\langle}
\def\r{\rangle}
\def\p{\partial}
\def\R{I\!\!R}
\def\12{\frac{1}{2}}
\def\sr1{1, \dots, r}
\def\a{\alpha}
\def\lmi{\lambda_{min}}
\def\lma{\lambda_{max}}
\def\ah{\hat{A}}
\def\bh{\hat{b}}
\def\xh{\hat{x}}
\def\yh{\hat{y}}
\def\ss{\sigma^*}
\def\s{\sigma}
\def\ab{\bar{A}}
\def\mb{\bar{M}}
\def\rb{\bar{R}}
\def\G{\Gamma}

%\useoutertheme[footline=empty]{shadow}
%\usecolortheme{crane}  %outer color
%\usecolortheme{rose}   %inner color
%\setbeamerfont{title}{shape=\itshape,family=\rmfamily}
%\setbeamercolor{title}{fg=red!80!black}
%\setbeamercolor{title}{fg=red!80!black,bg=red!20!white}

%culori activare/dezactivare
\def\colorize<#1>{                     
\temporal<#1>{\color{black}}{\color{black}}{\color{black!50}}}
%\usefonttheme{serif} % fontul serif is the best ...
% Titlu
\title
[Cuprins]
{\textbf{{\c Siruri}}}
\subtitle{}

%Autor
\author{\textbf{T\u anase Ramona Elena }}

%Institutia
\institute{Universitatea Ovidius Constanta\\
Facultatea de Matematic\u a \c si Informatic\u a\\
Specializarea:Matematic\u a - Informatic\u a}
%$^{\ast\ast}$Friedrich-Alexander Universit\"{a}t Erlangen-N\"{u}rnberg, Germany
%}           

%Data & informatii despre prezentare         
\date{
\textbf{
Iulie, 2021}\\
\bigskip
}

\begin{document}

\frame{\titlepage}
\frame
{
\frametitle{Cuprins}
\begin{itemize}
	\item[1.] \c Siruri
	\item[2.] \c Siruri convergente de numere reale 
	\item[3.] \c Siruri m\u arginite
	\item[4.] \c Siruri recurente \c si asimtote oblice
	\item[5.] Bibliografie
\end{itemize}
}
%%%%%%%%%%%%%%%%%%%%%%%%%%%%%%%%%%%%%%%%%%%%%%%%%%%%%%%%%%%%%%%%%%%%%%%%%%%%%%%%%%%%%%%%%%%%%%%%%%
\frame
{
\frametitle{\c Siruri}

\begin{definition}
Fie X o mulțime. O funcție \(f:\mathbb{N} \to X\) se numește șir de elemente din mulțimea X, sau sub o altă formulare: se numește șir de elemente din mulțimea X o funcție \(f:\mathbb{N} \to X\) . În mod uzual, se notează \(f_{1} = x_{1} \in X, f_{2} = x_{2} \in X,......, f_{n} = x_{n} \in X,....\)
\end{definition}

}
%%%%%%%%%%%%%%%%%%%%%%%%%%%%%%%%%%%%%%%%%%%%%%%%%%%%%%%%%%%%%%%%%%%%%%%%%%%%%%%%%%%%%%%%%%%%%%%%%%
\frame
{
\frametitle{\c Siruri m}
Teorie

Definiție

Un șir \((x_{n})_{n \in \mathbb{N}} \subset \mathbb{R}\) , se numește convergent dacă există \(x \in \mathbb{R}\) astfel încât:
	\(\forall _{\varepsilon } > 0, \in n_{\varepsilon } \in \mathbb{N}\) astfel încât este satisfacută inegalitatea:  \(\left | x_{n}- x \right | \leq \varepsilon\) . 
}
%%%%%%%%%%%%%%%%%%%%%%%%%%%%%%%%%%%%%%%%%%%%%%%%%%%%%%%%%%%%%%%%%%%%%%%%%%%%%%%%%%%%%%%%%%%%%%%%%%
\frame
{
\frametitle{\c Siruri convergente de numere reale}
Propoziție

Unicitatea limitei unui șir de numere reale

Fie \((x_{n})_{n \in \mathbb{N}} \subset \mathbb{R}\) . 

Dacă
\begin{displaymath}
  \left\{\begin{matrix}
x_{n} \to  x\\ 
x_{n} \to y
\end{matrix}\right.
\end{displaymath}

atunci \(x = y\).

}
%%%%%%%%%%%%%%%%%%%%%%%%%%%%%%%%%%%%%%%%%%%%%%%%%%%%%%%%%%%%%%%%%%%%%%%%%%%%%%%%%%%%%%%%%%%%%%%%%%
\frame
{
\frametitle{\c Siruri convergente de numere reale}
Demonstrație

Să presupunem, prin absurd, că \(x \neq  y\) . Cum suntem pe \(\mathbb{R}\) înseamnă că avem una din situațiile \(x < y\) sau \(y < x\). Pentru a face o alegere, fie \(x < y\) atunci \(y – x > 0\) și din definiție pentru \(\varepsilon = \frac{y- x}{2}  > 0\) rezultă că, 
\(\exists  n_{1} \in \mathbb{N}\) astfel încât \(\left | x_{n} - x  \right | < \frac{y - x }{2} , \forall n \geq n_{1} \)
și 
\(\exists  n_{2} \in \mathbb{N}\) astfel încât \(\left | x_{n} - y  \right | < \frac{y - x }{2} , \forall n \geq n_{2}\) 
Fie \(n = max (n _{1}, n_{2}) \geq n_{1}, n_{2}.\) Atunci: 
\(\left | x_{n} - x \right | < \frac{y-x}{2}\) și \(\left | x_{n} - y  \right | <  \frac{y-x}{2}\)
de unde 
\begin{displaymath}
  y-x = \left | y-x \right | = \left | (y-x_{n})+ (x_{n} -x) \right |\leq \left | y-x_{n} \right | + \left | x_{n} - x \right | < \frac{y-x}{2} + \frac{y-x}{2} = y-x
\end{displaymath}

	Așadar, \(y-x < y-x\) , contradicție!
}
%%%%%%%%%%%%%%%%%%%%%%%%%%%%%%%%%%%%%%%%%%%%%%%%%%%%%%%%%%%%%%%%%%%%%%%%%%%%%%%%%%%%%%%%%%%%%%%%%%
\frame
{
\frametitle{\c Siruri convergente de numere reale}
Exerciții

Calculați 

\begin{displaymath}
  \lim_{n\to\infty }\left ( \frac{1}{\sqrt{n^{4}+1}}+ \frac{2}{\sqrt{n^{4}+2} } +\frac{3}{\sqrt{n^{4}+3}}+........+\frac{n}{\sqrt{n^{4}+n}}  \right )
\end{displaymath}

 
Rezolvare

Notăm \(x_{n}= \frac{1}{\sqrt{n^{4}+1}} + \frac{2}{\sqrt{n^{4}+2}}+\frac{3}{\sqrt{n^{4}+3}}+........+\frac{n}{\sqrt{n^{4}+n}}\).
adică \(x_{n}= \sum_{k=1}^{n}\frac{k}{n^{4}+k}\).
 
În continuare procedăm astfel. De numărător nu ne atingem. Vom lucra cu numitorul, ideea fiind de a se avea același numitor peste tot. 
Avem \(1\leq k\leq n\) de unde \(n^{4}+1 \leq n^{4}+k \leq n^{4}+n\) de unde \(\sqrt{n^{4}+1}\leq \sqrt{n^{4}+k}\leq \sqrt{n^{4}+1}\) de unde \(\frac{1}{\sqrt{n^{4}+1}}\geq \frac{1}{\sqrt{n^{4}+k}}\geq \frac{1}{\sqrt{n^{4}+n}}\).
Acum înmulțind cu k obținem 


\(\frac{k}{\sqrt{n^{4}+1}}\geq \frac{k}{\sqrt{n^{4}+k}}\geq \frac{k}{\sqrt{n^{4}+n}}\)

În continuare în relația (1) dam lui k valorile \(1,2,.....,n\). 
}
%%%%%%%%%%%%%%%%%%%%%%%%%%%%%%%%%%%%%%%%%%%%%%%%%%%%%%%%%%%%%%%%%%%%%%%%%%%%%%%%%%%%%%%%%%%%%%%%%%
\frame
{
\frametitle{\c Siruri convergente de numere reale}

}
%%%%%%%%%%%%%%%%%%%%%%%%%%%%%%%%%%%%%%%%%%%%%%%%%%%%%%%%%%%%%%%%%%%%%%%%%%%%%%%%%%%%%%%%%%%%%%%%%%
\frame
{
\frametitle{\c Siruri m\u arginite}

}
%%%%%%%%%%%%%%%%%%%%%%%%%%%%%%%%%%%%%%%%%%%%%%%%%%%%%%%%%%%%%%%%%%%%%%%%%%%%%%%%%%%%%%%%%%%%%%%%%%
\frame
{
\frametitle{\c Siruri m\u arginite}

}
%%%%%%%%%%%%%%%%%%%%%%%%%%%%%%%%%%%%%%%%%%%%%%%%%%%%%%%%%%%%%%%%%%%%%%%%%%%%%%%%%%%%%%%%%%%%%%%%%%
\frame
{
\frametitle{\c Siruri m\u arginite}

}
%%%%%%%%%%%%%%%%%%%%%%%%%%%%%%%%%%%%%%%%%%%%%%%%%%%%%%%%%%%%%%%%%%%%%%%%%%%%%%%%%%%%%%%%%%%%%%%%%%
\frame
{
\frametitle{\c Siruri m\u arginite}

}
%%%%%%%%%%%%%%%%%%%%%%%%%%%%%%%%%%%%%%%%%%%%%%%%%%%%%%%%%%%%%%%%%%%%%%%%%%%%%%%%%%%%%%%%%%%%%%%%%%
\frame
{
\frametitle{\c Siruri recurente \c si asimtote oblice}

}
%%%%%%%%%%%%%%%%%%%%%%%%%%%%%%%%%%%%%%%%%%%%%%%%%%%%%%%%%%%%%%%%%%%%%%%%%%%%%%%%%%%%%%%%%%%%%%%%%%
\frame
{
\frametitle{\c Siruri recurente \c si asimtote oblice}

}
%%%%%%%%%%%%%%%%%%%%%%%%%%%%%%%%%%%%%%%%%%%%%%%%%%%%%%%%%%%%%%%%%%%%%%%%%%%%%%%%%%%%%%%%%%%%%%%%%%

\frame
{
\frametitle{Bibliografie}
\begin{itemize}
\item[(1)] Popa,Dumitru, Curs Matematică didactică – Analiză – Capitole speciale de analiză matematică pentru pregătirea profesoriloe , 2020-2021
\end{itemize}
}
\frame
{\frametitle{Multumiri}
\begin{center}
{\Large V\u a mul\c tumesc pentru aten\c tie !}	
\end{center}
}
%%%%%%%%%%%%%%%%%%%%%%%%%%%%%%%%%%%%%%%%%%%%%%%%%%%%%%%%%%%%%%%%%%%%%%%%%%%%%%%%%%%%%%%%%%%%%%%%%%
\end{document}
\documentclass[a4paper,12pt,oneside]{report}
\usepackage{OvidiusFMI}

\usepackage{times}
\usepackage{graphicx}
\usepackage{hyperref}
\usepackage{color,xcolor}
\usepackage{amsmath}
\usepackage{framed}
\usepackage{indentfirst}
\usepackage{enumerate}
\usepackage{listings}
\usepackage{amsmath,amsfonts,amssymb,amsthm,epsfig,epstopdf,url,array}
\usepackage{multicol,multirow}
\definecolor{code}{rgb}{0.97,0.97,0.97}
\lstdefinestyle{customc}{
  belowcaptionskip=1\baselineskip,
  backgroundcolor=\color{code},
  breaklines=true,
%  frame=L,
%  xleftmargin=\parindent,
  language=C,
  showstringspaces=false,
  morekeywords={bool,
  				 glutMainLoop, glutIdleFunc, glMatrixMode, glLoadIdentity, glPushMatrix, glPopMatrix, 
  				 glBegin, glEnd, glTranslatef, glRotatef, glScalef, glColor3f, glColor4f, glutSolidCube, glutWireCube, glutSolidSphere,
  				 glutWireSphere, glutSolidCone,glutSetWindowTitle,glutGet,glClear,glutSwapBuffers,glDepthFunc,
  				 glutWireCone, glutSolidTorus, glutWireTorus, glutSolidDodecahedron, glutWireDodecahedron,
  				 glutSolidOctahedron, glutWireOctahedron, glutSolidTetrahedron, glutWireTetrahedron, 
  				 glVertex3f,glVertex2f,glPointSize,
  				 glutSolidIcosahedron, glutWireIcosahedron, glutSolidTeapot, glutWireTeapot,glutReshapeFunc,
  				 glFlush, gluPerspective, glutPostRedisplay, glutInit, glutKeyboardFunc,glutKeyboardUpFunc,
  				 glutInitWindowSize, glutInitWindowPosition, glutInitDisplayMode, glutCreateWindow, glutDisplayFunc,glutPassiveMotionFunc,
  				 glClear,glTexCoord2f,
  				 glEnable, glDisable, glLightfv, glMaterialfv, glCullFace,glViewport,
  				 glFrontFace,glColor3ub, glShadeModel,
  				 glGenLists, glGetFloatv,glGentextures,glTexImage2D,glTexParameteri, free, glDeleteTextures,
  				 glLineStipple, glLineWidth, glBindTexture,glGenTextures,
  				 glNewList, glEndList, glCallList,
  				 glMap1f,glEvalCoord1f,glMapGrid1d,glEvalMesh1,glMap2f,glEvalCoord2f,glMapGrid2f,glEvalMesh2,
  				 gluBeginTrim, gluEndTrim, gluPwlCurve,glHint,
  				 GLUnurbsObj, gluBeginSurface, gluNurbsSurface, gluEndSurface, gluNewNurbsRenderer, 									gluNurbsProperty,gluQuadricNormals,
  				 glNormal3f,
  				 gluQuadricTexture,GLUquadricObj,gluSphere,
  				 glPolygonMode, glBlendFunc,glFogi,glFogiv,glFogfv,
  				 GLfloat, GLdouble, GLint,GLuint, GLushort,GLubyte, glRasterPos2f,
  				 gluBeginCurve, gluNurbsCurve, gluEndCurve,
  				 glOrtho, gluLookAt, glutBitmapCharacter, 
  				 glInitNames, glPushName, glLoadName, glSelectBuffer, glRenderMode,gluPickMatrix, glGetIntegerv, glutMouseFunc,glutMotionFunc,system,
  				 glPushAttrib, glPopAttrib, glMultMatrixf, sprintf, glClearStencil, glStencilFunc,glStencilOp,glStencilMask,glColorMask,glActiveStencilFaceEXT,fprintf},  
%   numbers=left,                    % where to put the line-numbers; possible values are (none, left, right)
  %numbersep=5pt,                   % how far the line-numbers are from the code
  %numberstyle=\tiny\color{code}, % the style that is used for the line-numbers
  basicstyle=\footnotesize\ttfamily,
  keywordstyle=\bfseries\color{green!40!black},
  commentstyle=\itshape\color{purple!40!black},
  identifierstyle=\color{blue},
  stringstyle=\color{orange},
}

\lstset{escapechar=@,style=customc}


\facultatea{Matematic\u a \c si Informatic\u a}
\specializarea{Informatic\u a}
\teza{licen\c t\u a}
\titlu{\c Siruri}
\coordonatorPrincipal{} %OBLIGATORIU
\autor{Tănase Ramona Elena}
\data{2021}

\begin{document}
\maketitle

\pagenumbering{roman}
\tableofcontents

\pagenumbering{arabic}


\chapter{\c Siruri}

\textbf{Defini\c tie}

Fie X o mulțime. O funcție \(f:\mathbb{N} \to X\) se numește șir de elemente din mulțimea X, sau sub o altă formulare: se numește șir de elemente din mulțimea X o funcție \(f:\mathbb{N} \to X\). În mod uzual, se notează \(f_{1} = x_{1} \in X, f_{2} = x_{2} \in X,......, f_{n} = x_{n} \in X,....\)

\chapter{\c Siruri convergente de numere reale}


\textbf{Defini\c tie}

Un șir \((x_{n})_{n \in \mathbb{N}} \subset \mathbb{R} \), se numește convergent dacă există \(x \in \mathbb{R}\) astfel încât:
\(\forall _{\varepsilon } > 0, \in n_{\varepsilon } \in \mathbb{N} \) astfel încât este satisfacută inegalitatea: \(\left | x_{n}- x \right | \leq \varepsilon \).

\textbf{Propozi\c tie}

Unicitatea limitei unui șir de numere reale
Fie \((x_{n})_{n \in \mathbb{N}} \subset \mathbb{R}\). 
Dacă \(\left\{\begin{matrix}
x_{n} \to  x\\ 
x_{n} \to y
\end{matrix}\right.
\) atunci \(x=y.\)

\textbf{Demonstra\c tie}

Să presupunem, prin absurd, că \(x \neq  y\). Cum suntem pe \(\mathbb{R}\) înseamnă că avem una din situațiile \(x < y\) sau \(y < x\). Pentru a face o alegere, fie \(x < y\) atunci \(y – x > 0\) și din definiție pentru \(\varepsilon = \frac{y- x}{2}  > 0\) rezultă că, 

\begin{itemize}
  \item \(\exists  n_{1} \in \mathbb{N}\) astfel încât \(\left | x_{n} - x  \right | < \frac{y - x }{2} , \forall n \geq n_{1} \)
  \item \(\exists  n_{2} \in \mathbb{N}\) astfel încât \(\left | x_{n} - y  \right | < \frac{y - x }{2} , \forall n \geq n_{2} \)
\end{itemize}

Fie \(n = max (n _{1}, n_{2}) \geq n_{1}, n_{2}.\) Atunci \(\left | x_{n} - x \right | < \frac{y-x}{2} și \left | x_{n} - y  \right | <  \frac{y-x}{2}\) de unde 

\begin{displaymath}
  y-x = \left | y-x \right | = \left | (y-x_{n})+ (x_{n} -x) \right |\leq \left | y-x_{n} \right | + \left | x_{n} - x \right | < \frac{y-x}{2} + \frac{y-x}{2} = y-x
\end{displaymath}


Așadar, \(y-x < y-x\), contradicție!

Un rezultat foarte frecvent folosit este ceea ce se numește ”teorema cleștelui”.

\textbf{Teorema cleștelui}

Fie \((x_{n})_{n\in \mathbb{N}}, (y_{n})_{n\in \mathbb{N}},(z_{n})_{n\in \mathbb{N}} \) trei șiruri de numere reale. 
Dacă:

\[\left\{\begin{matrix}
x_{n} \leq  y_{n} \leq z_{n}, \forall  n \in \mathbb{N}\\ 
x_{n} \to x, z_{n} \rightarrow x

\end{matrix}\right. \]
Atunci \(y_{n} \to x\).

\textbf{Demonstra\c tie}

Vom arăta pentru început următoarea inegalitate. Dacă \(a \leq x\leq b\) atunci \(\left | x \right | \leq  max (\left | a \right |, \left | b \right |) \). 
Vom folosi proprietățile de la modul. Avem:

\begin{displaymath}
  \left | x \right | = \left\{\begin{matrix} x, daca x \geq 0\\ -x, daca x< 0 \end{matrix}\right. \leq \left\{\begin{matrix} b\leq max(b,-b) = \left | b \right |\leq max(\left | a \right |,\left | b \right |) daca x\geq 0\\  -a\leq max(a,-a) = \left | a \right | \leq max (\left | a \right |,\left | b \right |) daca x< 0 \end{matrix}\right. \leq max (\left | a \right |, \left | b \right |)
\end{displaymath}
 

Din \(x_{n} \leq y_{n}\leq z_{n}\), \(\forall n\in \mathbb{N} \) rezultă că \(x_{n}-x \leq y_{n}-x \leq z_{n}-x, \forall n\in \mathbb{N}. \)
De aici folosind inegalitatea demonstrată deducem că:

%TODO
\[\left | y_{n} - x \right |\leq max (\left | x_{n}-x \right |, \left | z_{n} - x \right |),\forall n\in \mathbb{N} \tag{1.1}\label{eq:1.1}\]


Deoarece \(x_{n} \to x, \forall \varepsilon > 0, \exists {n_{\varepsilon }}'\in \mathbb{N}\) astfel încât pentru \(\forall n \geq {n_{\varepsilon }}'\) este satisfacută inegalitatea \[\left | x_{n}-x \right |< \varepsilon.\tag{1.2}\label{eq:1.2} \]
	
Similar din \(z_{n} \to x, \forall \varepsilon > 0,\exists {n_{\varepsilon }}'' \in \mathbb{N}\) astfel încât pentru \(\forall n\geq {n_{\varepsilon }}''\) este satisfacută inegalitatea \[\left | z_{n} -x \right |< \varepsilon. \tag{1.3}\label{eq:1.3}\]

Fie acum \(\varepsilon > 0.\) Notăm \(n_{\varepsilon } = max ({n_{\varepsilon }}', {n_{\varepsilon }}'')\). Fie acum \(n\geq n_{\varepsilon }\). Deoarece \(n_{\varepsilon \geq }{n_{\varepsilon }}'\) iar \(n\geq n_{\varepsilon }\) rezultă că \(n\geq {n_{\varepsilon }}'\) și din \eqref{eq:1.2} rezultă că \[\left | x_{n} - x \right |< \varepsilon \tag{1.4}\label{eq:1.4} \]
	
Deoarece \(n_{\varepsilon }\geq {n}''_{\varepsilon }\) iar \(n\geq n_{\varepsilon}\) și din \eqref{eq:1.3} rezultă că  \[\left | z_{n}-x  \right |< \varepsilon \tag{1.5}\label{eq:1.5}\]
	
Din \eqref{eq:1.4} și \eqref{eq:1.5} rezultă că

\[ max(\left | x_{n} -x  \right |, \left | z_{n}-x \right |) = \begin{Bmatrix}
\left | x_{n}-x daca  \right |\\ 
\left | z_{n}-x daca  \right |
\end{Bmatrix} < \varepsilon.  \tag{1.6}\label{eq:1.6} \]

Folosind inegalitatea \eqref{eq:1.6} din inegalitatea \eqref{eq:1.1} deducem că \(\left | y_{n}-x  \right |< \varepsilon\).
 
Așadar am demonstrat : \(\forall \varepsilon > 0, \exists n_{\varepsilon \in \mathbb{N}}\) astfel încât pentru \(\forall n\geq n_{\varepsilon }\) este satisfacută inegalitatea \(\left | y_{n}-x \right |< \varepsilon.\) 

Conform definiției această inegalitate înseamnă că \(y_{n} \to y.\) 

\textbf{Exemplu}

Fie \(c \in \mathbb{R}\), Considerăm șirul \(x_{n}=c\). Atunci \(\lim_{n \to \infty }x_{n}=c\) sau \(\lim_{n \to \infty }c=c\), limita unei constante este acea constantă. 

\textbf{Demonstrație}

\(\forall n\in \mathbb{N}\) avem \(x_{n} - c = c - c = 0 , \left | x_{n}-c \right |= 0\). De aici deducem că \(\forall \varepsilon > 0, \exists n_{\varepsilon} = 1 \in \mathbb{N}\) astfel încât pentru \(\forall n\geq n_{\varepsilon }= 1\) este satisfacută inegalitatea \(\left | x_{n}-c \right |= 0< \varepsilon\). 
	Conform definiției \(\lim_{n \to \infty }x_{n} = c. \)
	
\textbf{Propoziție}

Dacă un șir de numere naturale este convergent atunci el este staționar. 
Fie \((x_{n})_{n\in \mathbb{N}}\) un șir de numere naturale. Dacă există \(x\in \mathbb{R}\) astfel încât \(\lim_{n \to \infty }x_{n}= x\), atunci există \(k\in \mathbb{N}\) astfel încât \(x_{n}= x_{k}, \forall n\geq k\).
	
Astfel spus scris desfășurat șirul arată astfel:
\begin{displaymath}
x_{1},x_{2},x_{3},x_{4},.........,x_{k-1},x_{k},x_{k},x_{k}........
\end{displaymath}




\textbf{Demonstrație}

Deoarece \(\lim_{n \to \infty }x_{n}= x\) pentru \(\varepsilon = \frac{1}{2}> 0, \exists n_{\frac{1}{2}}\in \mathbb{N}\) astfel încât \(\forall n\geq n_{\frac{1}{2}}\) este satisfacută inegalitatea \(\left | x_{n} -x \right |<  \frac{1}{2}\). 
	
Să notăm \(k=n_{\frac{1}{2}}\in \mathbb{N}\) și să reținem că știm că \(\forall n\geq k \) este satisfacută inegalitatea 

%TODO
\begin{displaymath}
  \left | x_{n} -x \right |< \frac{1}{2}. (1) 
\end{displaymath}


Fie \(n\geq k\). Relația (1) fiind adevărată pentru orice număr \(\geq k\) ea va fi adevărată în particular pentru k adică avem 
\begin{displaymath}
  \left | x_{k}-x \right |< \frac{1}{2}. (2)
\end{displaymath}


Dar la noi \(n\geq k\) deci din (1) avem și 
\begin{displaymath}
  \left | x_{n}-x \right |< \frac{1}{2}.(3)
\end{displaymath}


Avem 
\begin{displaymath}
  \left | x_{n}-x_{k} \right |= \left | (x_{n}-x)+(x-x_{k}) \right |\leq \left | x_{n}-x \right |+\left | x-x_{k} \right |= 
\end{displaymath}
\begin{displaymath}
  =\left | x_{n}-x \right |+ \left | -(x-x_{k}) \right |= \left | x_{n}-x \right |+ \left | x_{k} -x\right |. (4)
\end{displaymath}



Am folosit inegalitatea tringhiului și \(\left | -a \right |= \left | a \right |\). Folosind (2) și (3) din (4) deducem că 
\begin{displaymath}
  \left | x_{n}-x_{k} \right |< \frac{1}{2}+ \frac{1}{2}= 1. (5) 
\end{displaymath}


Dar \(x_{n}, x_{k}\) sunt numere naturale, și deci diferența lor este un număr întreg adică \(x_{n}- x_{k}\in \mathbb{Z}\). Cum \(\left |x_{n}- x_{k} \right |\geq 0\) iar din (5) \(\left |x_{n}- x_{k} \right |< 1\) rezultă că \(\left |x_{n}- x_{k} \right |\in \left [ 0,1 \right ]\) deci \(\left |x_{n}- x_{k} \right |\in\mathbb{Z}\cap \left [ 0,1 \right)= \left \{ ....,-n ,....,-2,-1,0,1,2,3,....,n,... \right \}\cap \left [ 0,1 \right )= \left \{ 0 \right \}\) de unde \(\left | x_{n}-x_{k} \right |=0\) adică \(x_{n}-x_{k}=0,x_{n}=x_{k}.\) 

Așasar am demonstrat: \(\forall n\geq k avem x_{n}=x_{k},\) ceea ce încheie demonstrația. 

\section{Exerciții}

\begin{enumerate}
 \item Calculați
\begin{displaymath}
   \lim_{n\to\infty }\left ( \frac{1}{\sqrt{n^{4}+1}}+ \frac{2}{\sqrt{n^{4}+2} } +\frac{3}{\sqrt{n^{4}+3}}+........+\frac{n}{\sqrt{n^{4}+n}}  \right )
\end{displaymath}


\textbf{Rezolvare}

Notăm \( x_{n}= \frac{1}{\sqrt{n^{4}+1}} + \frac{2}{\sqrt{n^{4}+2}}+\frac{3}{\sqrt{n^{4}+3}}+........+\frac{n}{\sqrt{n^{4}+n}} \).
Adică \( x_{n}= \sum_{k=1}^{n}\frac{k}{n^{4}+k}\).

În continuare procedăm astfel. De numărător nu ne atingem. Vom lucra cu numitorul, ideea fiind de a se avea același numitor peste tot. 

Avem 
\begin{displaymath}
  1\leq k\leq n \Rightarrow 
\end{displaymath}
\begin{displaymath}
  \Rightarrow  n^{4}+1 \leq n^{4}+k \leq n^{4}+n \Rightarrow 
\end{displaymath}
\begin{displaymath}
  \Rightarrow  \sqrt{n^{4}+1}\leq \sqrt{n^{4}+k}\leq \sqrt{n^{4}+1} \Rightarrow 
\end{displaymath}
\begin{displaymath}
  \Rightarrow  \frac{1}{\sqrt{n^{4}+1}}\geq \frac{1}{\sqrt{n^{4}+k}}\geq \frac{1}{\sqrt{n^{4}+n}}.
\end{displaymath}


Acum înmulțind cu k obținem \(\frac{k}{\sqrt{n^{4}+1}}\geq \frac{k}{\sqrt{n^{4}+k}}\geq \frac{k}{\sqrt{n^{4}+n}}\). (1) 

În continuare în relația (1) dam lui \(k\) valorile \(1,2,.....,n\). 

Pentru \(k = 1\) rezultă:

\begin{displaymath}
  \frac{1}{\sqrt{n^{4}+1}}\geq \frac{1}{\sqrt{n^{4}+k}}\geq \frac{1}{\sqrt{n^{4}+n}} 
\end{displaymath}


Pentru \(k = 2\) rezultă:

\begin{displaymath}
  \frac{2}{\sqrt{n^{4}+1}}\geq \frac{2}{\sqrt{n^{4}+2}}\geq \frac{2}{\sqrt{n^{4}+n}}
\end{displaymath}


Adunând inegalitățile de mai sus obținem 

\begin{displaymath}
  \frac{1}{\sqrt{n^{4}+1}}+ \frac{2}{\sqrt{n^{4}+1}}+......+ \frac{n}{\sqrt{n^{4}+1}} \geq \frac{1}{\sqrt{n^{4}+1}}+ \frac{2}{\sqrt{n^{4}+2}}+......+ \frac{n}{\sqrt{n^{4}+n}}\geq
\end{displaymath}
\begin{displaymath}
  \geq\frac{1}{\sqrt{n^{4}+n}}+ \frac{2}{\sqrt{n^{4}+n}}+......+ \frac{n}{\sqrt{n^{4}+n}}
\end{displaymath}


Sau
\begin{displaymath}
  \frac{1+2+....+n}{\sqrt{n^{4}+1}}\geq x_{n}\geq \frac{1+2+.....+n}{\sqrt{n^{4}+n}}
\end{displaymath}


Dar știm că \(1+2+...+n = \frac{n(n+1)}{2}\), deci vom obține 
\begin{displaymath}
  \frac{n(n+1)}{2\sqrt{n^{4}+1}}\geq x_{n}\geq \frac{n(n+1)}{2\sqrt{n^{4}+n}} (2)
\end{displaymath}

Acum 
\begin{displaymath}
  \lim_{n \to \infty }\frac{n(n+1)}{2\sqrt{n^{4}+1}}=\frac{1}{2} și \lim_{n \to \infty }\frac{n(n+1)}{2\sqrt{n^{4}+n}}=\frac{1}{2} (3)
\end{displaymath}


Vom da la ambele factor comun forțat. 

Din (2) și (3) și teorema cleștelui rezultă că:
\begin{displaymath}
  \lim_{n \to \infty }x_{n}=\frac{1}{2}
\end{displaymath}
\end{enumerate}


\chapter{Șiruri mărginite}

\textbf{Definiție}

Fie \((x_{n})_{n\in \mathbb{N}}\) un șir de numere reale. Șirul \((x_{n})_{n\in \mathbb{N}}\) se numește mărginit dacă și numai dacă \(\exists  a, b \in \mathbb{R}, a< b\) astfel încât \(\forall n\in \mathbb{N}\) este satisfacută inegalitatea \(x_{n}\in \left [ a,b \right ]\), sau echivalent \(\exists M> 0\) astefle încât \(\forall  n\in \mathbb{N}\) este satisfacută inegalitatea \(\left | x_{n} \right |\leq M\).

\textbf{Definiție }

Fie \((x_{n})_{n\in \mathbb{N}}\) un șir de numere reale. Spunem că \(\lim_{n \to \infty }x_{n}=\infty\) dacă, \(\forall \varepsilon > 0,\exists n_{\varepsilon }\in \mathbb{N}\) astfel încât pentru \(\forall n\geq n_{\varepsilon }\) este satisfacută inegalitatea \(x_{n}> \varepsilon\). 
Sau \(\forall \varepsilon > 0,\exists n_{\varepsilon }\in \mathbb{N}\) astfel încât \(x_{n}> \varepsilon ,\forall n\geq n_{\varepsilon }\). 

\textbf{Propoziție}
 
Fie \((x_{n})_{n\in \mathbb{N}}\) un șir de numere reale. Dacă \(\lim_{n \to \infty }x_{n}=\infty\) atunci \( \lim_{n \to \infty }\frac{1}{x_{n}}=0.\) 

\textbf{Demonstrație} 

Fie \(\varepsilon > 0\). Deoarece \(\lim_{n \to \infty }x_{n}=\infty\) din definiție aplicată pentru \(\frac{1}{\varepsilon }> 0\) rezultă că \(\exists n_{\varepsilon }\in \mathbb{N}\) astfel încât pentru \(\forall n\geq n_{\varepsilon }\) este satisfacută inegalitatea \(x_{n}> \frac{1}{\varepsilon }\). 

Din această inegalitate rezultă că \(\forall n\geq n_{\varepsilon }\) este satisfacută inegalitatea \(x_{n}> 0\), prin urmare are sens fracția \(\frac{1}{x_{n}}, \forall n\geq n_{\varepsilon }\). Dar inegalitatea de mai sus este echivalentă cu \(\exists n_{\varepsilon }\in \mathbb{N}\) astfel încât \(\forall n\geq n_{\varepsilon }\) este satisfacută inegalitatea\( \frac{1}{x_{n}}< \varepsilon.\) Conform definiției aceasta înseamnă că \(\lim_{n \to \infty }\frac{1}{x_{n}}=0\).


\textbf{Lema Stolz-Cesaro (Cazul \(\frac{1}{\infty }\))}

Fie \(\left ( x_{n} \right )_{n\in \mathbb{N}}\subset \mathbb{R}\) și \(\left (\alpha _{n} \right )_{n\in \mathbb{N}}\subset \left ( 0,\infty \right )\) astfel încât \(\alpha _{n} \uparrow \infty\). 
Dacă 
\begin{displaymath}
  \lim_{n \to \infty }\frac{x_{n} - x_{n-1}}{\alpha _{n}-a_{n-1}}\in \mathbb{R}
\end{displaymath}
	atunci 
\begin{displaymath}
  \lim_{n \to \infty }\frac{x_{n}}{ \alpha _{n}}\in \mathbb{R} 
\end{displaymath}
	și în plus 
\begin{displaymath}
  \lim_{n \to \infty }\frac{x_{n}}{ \alpha _{n}}= \lim_{n \to \infty }\frac{x_{n} - x_{n-1}}{ \alpha _{n}- \alpha _{n-1}}.
\end{displaymath}


\textbf{Demonstrație} 
Fie \(\alpha = \lim_{n \to \infty }\frac{x_{n}-x_{n-1}}{\alpha _{n}-\alpha _{n-1}}\) . 
Atunci \(\forall \varepsilon > 0,\exists n_{\varepsilon }\in \mathbb{N}\) astefl încât \(\left | \frac{x_{n}-x_{n-1}}{\alpha _{n}- \alpha _{n-1}} - \alpha \right |< \frac{\varepsilon }{2} \forall  n\geq n_{\varepsilon  }\)
Sau , \(\alpha _{n} \uparrow, 
\left | x_{n}- x_{n-1 }- \alpha \left ( \alpha _{n}-\alpha _{n-1} \right ) \right | < \frac{\varepsilon }{2}\left ( \alpha _{n}- \alpha _{n-1} \right ), \forall n\geq n_{\varepsilon }\). (1) 

Notăm cu \(k=n_{\varepsilon }+1\). Pentru \(n\geq k \)luând în (1), \(n= k+1, k+2,....,n\) obținem:
\(\left | x_{k+1} - x_{k} - \alpha \left ( a_{k+1}- a_{k} \right ) \right |< \frac{\varepsilon }{2}\left ( \alpha _{k+1} - \alpha _{k} \right )\). 

\(\left | x_{k+2} - x_{k+1} - \alpha \left ( a_{k+2}- a_{k+1} \right ) \right |< \frac{\varepsilon }{2}\left ( \alpha _{k+2} - \alpha _{k+1} \right )
...
...
...
\left | x_{n} - x_{n-1} - \alpha \left ( a_{n}- a_{n-1} \right ) \right |< \frac{\varepsilon }{2}\left ( \alpha _{n} - \alpha _{n-1} \right )\)

De unde obținem, prin adunare:

\begin{displaymath}
  \left | x_{n} - x_{k} - \alpha \left ( \alpha _{n}-\alpha _{k} \right ) \right |= 
\end{displaymath}
\begin{displaymath}
  \left | x_{n} - x_{n-1} -\alpha \left ( \alpha _{n}-\alpha _{n-1} \right )+.....+x_{k+2}-x_{k+1}-\alpha \left ( \alpha _{k+2}-\alpha _{k+1} \right )  +x_{k+1}-x_{k}-\alpha \left ( \alpha _{k+1}-\alpha _{k} \right )\right |
\end{displaymath}
\begin{displaymath}
  \leq \left | x_{n}-x_{n-1}-\alpha \left ( \alpha _{n}-\alpha _{n-1} \right ) \right |+.......+\left | x_{k+2}-x_{k+1}-\alpha \left ( \alpha _{k+2}-\alpha _{k+1} \right ) \right |+\left | x_{k+1}-x_{k}- \alpha \left ( \alpha _{k+1}-\alpha _{k} \right ) \right |
\end{displaymath}
\begin{displaymath}
  \leq \frac{\varepsilon }{2}\left ( \alpha _{k+1} -\alpha _{k} \right )+\frac{\varepsilon }{2}\left ( \alpha _{k+2}-\alpha _{k+1} \right )+\frac{\varepsilon }{2}\left ( \alpha _{n}- \alpha _{n-1}\right )= \frac{\varepsilon }{2}\left ( \alpha _{n}-\alpha _{k} \right )
\end{displaymath}

\(\leq \frac{\varepsilon }{2}\alpha _{n}\) deoarece \(\alpha _{k}> 0\). 


\section{Exerciții}

\begin{enumerate}
\item Calculați 

 Fie \(\alpha > 0\) să se calculeze 

\begin{displaymath}
  \lim_{n \to \infty }\frac{1^{\alpha }+2^{\alpha }+....+n^{\alpha }}{n^{\alpha +1}}
\end{displaymath}


\textbf{Demonstrație} 

Fie \(x_{n}=1^{\alpha }+2^{\alpha }+....+n^{\alpha },a_{n}= n^{\alpha }\). Deoarece \(\alpha > 0 , \alpha \uparrow \infty.\) Din lema Stolz-Cesaro, cazul 
\begin{displaymath}
  \left [ \frac{1}{\infty } \right ], 
\lim_{n \to \infty }\frac{1^{\alpha }+2^{\alpha }+....+n^{\alpha }}{n^{\alpha +1}}=\lim_{n \to \infty }\frac{x_{n}}{\alpha _{n}}= \lim_{n \to \infty } \frac{x_{n+1}-x_{n}}{\alpha _{n+1}-_{n}}=\lim_{n \to \infty } \frac{\left ( n+1 \right )^{\alpha }}{\left ( n+1 \right )^{\alpha+1} -n^{\alpha +1}}
\end{displaymath}
\begin{displaymath}
  \lim_{n \to \infty }\frac{\left ( n+1 \right )^{\alpha +1}-n^{\alpha +1}}{\left ( n+1 \right )^{\alpha }}
\end{displaymath}

Dăm factor comun forțat la numărător pe \(n^{\alpha +1}\). Avem 
\begin{displaymath}
  \lim_{n \to \infty }\frac{\left ( n+1 \right )^{\alpha +1}-n^{\alpha +1}}{\left ( n+1 \right )^{\alpha }} = \lim_{n \to \infty }\frac{n^{\alpha +1\left [ \frac{\left ( n+1 \right )^{\alpha +1}}{n^{\alpha +1}} -1\right ]}}{\left ( n+1 \right )^{\alpha }} 
\end{displaymath}
\begin{displaymath}
  = \lim_{n \to \infty }\frac{n^{\alpha }}{\left ( n+1 \right )^{\alpha }}\cdot n\left [ \left ( \frac{n+1}{n} \right )^{\alpha +1}-1 \right ]
\end{displaymath}
\begin{displaymath}
  =\lim_{n \to \infty }\frac{n^{\alpha }}{\left ( n+1 \right )^{\alpha }}\cdot \lim_{n \to \infty }n\left [ \left ( \frac{n+1}{n} \right )^{\alpha +1} -1\right ]
\end{displaymath}
\begin{displaymath}
  =\lim_{n \to \infty }n\left [ \left ( 1+\frac{1}{n} \right )^{\alpha +1}-1 \right ]
\end{displaymath}
\begin{displaymath}
  = \lim_{n \to \infty }\frac{\left ( 1+\frac{1}{n} \right )^{\alpha +1}-1}{\frac{1}{n}}
\end{displaymath}
\begin{displaymath}
  =\lim_{n \to \infty }\frac{\left ( 1+n \right )^{\alpha +1}-1}{n}= \alpha +1
\end{displaymath}

Am folosit limita fundamentală 

\begin{displaymath}
  \lim_{x \to \infty }\frac{\left ( 1+x \right )^{\gamma }-1}{x} = \gamma ,\gamma \in \mathbb{R}
\end{displaymath}


Întorcându-ne la problemă, obținem:

\begin{displaymath}
  \lim_{n \to \infty }\frac{1^{\alpha }+2^{\alpha }+....+n^{\alpha }}{n^{\alpha +1}} = \frac{1}{\alpha +1}
\end{displaymath}


\item Calculați
\begin{displaymath}
  \lim_{n \to \infty }\frac{e^{\sqrt{1}}+ e^{\sqrt{2}}+....+e^{\sqrt{n}}}{\sqrt{n}e^{\sqrt{n}}}
\end{displaymath}


\textbf{Rezolvare}

Fie \( x_{n} = e^{\sqrt{1}}+ e^{\sqrt{2}}+......+e^{\sqrt{n}}, a_{n}= \sqrt{n}e^{\sqrt{n}}.\) Avem de calculat \(\lim_{n \to \infty }\frac{x_{n}}{a_{n}}= \left [ \frac{1}{\infty } \right ]\)


Din lema Stolz-Cesaro 


\begin{displaymath}
  \lim_{n \to \infty }\frac{x_{n}}{a_{n}}= \lim_{n \to \infty }\frac{x_{n+1}-x_{n}}{\alpha _{n+1}-\alpha _{n}}=\lim_{n \to \infty }\frac{e^{\sqrt{n+1}}}{\sqrt{n+1}e^{\sqrt{n+1}}-\sqrt{n}e^{\sqrt{n}}}
\end{displaymath}


Vom calcula acum 
\begin{displaymath}
  \lim_{n \to \infty }\frac{\sqrt{n+1}e^{\sqrt{n+1}}-\sqrt{n}e^{\sqrt{n}}}{e^{\sqrt{n+1}}}
\end{displaymath}


Avem 
\begin{displaymath}
  \lim_{n \to \infty }\frac{\sqrt{n+1}e^{\sqrt{n+1}}-\sqrt{n}e^{\sqrt{n}}}{e^{\sqrt{n+1}}} = \lim_{n \to \infty }\frac{\left ( \sqrt{n+1}-\sqrt{n} \right )e^{\sqrt{n+1}}+ \sqrt{n}\left ( e^{\sqrt{n+1}}-e^{\sqrt{n}} \right )}{e^{\sqrt{n+1}}}
\end{displaymath}

\begin{displaymath}
  = \lim_{n \to \infty }\left ( \sqrt{n+1}-\sqrt{n} \right )+ \lim_{n \to \infty }\frac{\sqrt{n}\left ( e^{\sqrt{n+1}} -e^{\sqrt{n}} \right )}{e^{\sqrt{n+1}}} 
\end{displaymath}


Prima limită, înmulțind și împărțind cu conjugata ei ne da da 0, adică:
\begin{displaymath}
  \lim_{n \to \infty }\left ( \sqrt{n+1}-\sqrt{n} \right ) = 0
\end{displaymath}


Pentru cea de a doua limită procedăm astfel:
\begin{displaymath}
  \lim_{n \to \infty }\frac{\sqrt{n}\left ( e^{\sqrt{n+1}} -e^{\sqrt{n}} \right )}{e^{\sqrt{n+1}}}  = \lim_{n \to \infty } \sqrt{n}\left ( 1- \frac{e^{\sqrt{n}}}{e^{\sqrt{n+1}}} \right )
\end{displaymath}

\begin{displaymath}
  =\lim_{n \to \infty }\sqrt{n}\left ( 1-e^{\sqrt{n}-\sqrt{n+1}} \right )
\end{displaymath}

\begin{displaymath}
  = - \lim_{n \to \infty } \frac{e^{\sqrt{n}- \sqrt{n+1}}-1}{\sqrt{n}- \sqrt{n+1}} \cdot \sqrt{n}\left ( \sqrt{n}-\sqrt{n+1} \right )
\end{displaymath}

\begin{displaymath}
  = -\lim_{n \to \infty } \frac{e^{\sqrt{n}- \sqrt{n+1}}-1}{\sqrt{n}- \sqrt{n+1}} \cdot \lim_{n \to \infty }\sqrt{n}\left ( \sqrt{n} -\sqrt{n+1}\right )
\end{displaymath}

\begin{displaymath}
  = -1 \cdot \left ( -\frac{1}{2} \right ) =\frac{1}{2}
\end{displaymath}

Am înmulțit și am împărțit cu conjugata ei, iar la ultima factor comun forțat. 

Din acestea deducem că:
\begin{displaymath}
  \lim_{n \to \infty }\frac{e^{\sqrt{n+1}}}{\sqrt{n+1}e^{\sqrt{n+1}}-\sqrt{n}e^{\sqrt{n}}} = 2
\end{displaymath}


Deci ultima limită din enunț \(\lim_{n \to \infty } \frac{x_{n}}{\alpha _{n}} = 2\)
\end{enumerate}

\textbf{Propoziție} 

Fie \(\left ( x_{n} \right )_{n\in \mathbb{N}}\) un șir de numere reale strict pozitive. Dacă \(\lim_{n \to \infty }\frac{x_{n+1}}{x_{n}}\in \mathbb{R}\) atunci \(\lim_{n \to \infty } \sqrt[n]{x_{n}}\in \mathbb{R}\). În plus \(\lim_{n \to \infty } \sqrt[n]{x_{n}} = \lim_{n \to \infty }\frac{x_{n+1}}{x_{n}}\). 
Pe scurt 

\begin{displaymath}
  \lim_{n \to \infty } \sqrt[n]{x_{n}} = \lim_{n \to \infty }\frac{x_{n+1}}{x_{n}}
\end{displaymath}



\textbf{Demonstrație} 

Din definiția logaritmilor naturali avem 
\begin{displaymath}
  \ln x = \alpha  \Leftrightarrow x = e^{\alpha }
\end{displaymath}

De aici deducem că \(x = e^{\alpha } = e^{\ln x}\) adică 
\begin{displaymath}
  x = e^{\ln x}, \forall x> 0
\end{displaymath}

De aici dacă \(x=u^{v}\) obținem \(u^{v} = e^{\ln\left ( u^{v} \right )} = u^{v\ln u}\). 
Să reținem această egalitate
\begin{displaymath}
  u^{v} = u^{v\ln u}
\end{displaymath}

Ea se folosește tot timpul când baza și puterea sunt variabile. La noi  \(\sqrt[n]{x_{n}}= x_{n}^{\frac{1}{n}} \)de unde folosind egalitatea de mai sus rezultă că 
\begin{displaymath}
  \sqrt[n]{x_{n}}= x_{n}^{\frac{1}{n}} = e^{\frac{1}{n}\cdot \ln x_{n}} = e^{\frac{\ln x_{n}}{n}}
\end{displaymath}


Fie \(A = \lim_{n \to \infty }\frac{x_{n}+1}{x_{n}}\). Atunci \(\ln\) fiind funcție continuă 
\begin{displaymath}
  \ln A = \ln \lim_{n \to \infty }\frac{x_{n}+1}{x_{n}} = \lim_{n \to \infty }\ln \frac{x_{n}+1}{x_{n}}.
\end{displaymath}


Vom arăta că în ipotezele noastre 
\begin{displaymath}
  \lim_{n \to \infty }\frac{\ln x_{n}}{n} = \lim_{n \to \infty }\ln\frac{x_{n+1}}{x_{n}}
\end{displaymath}

 Din lema Stolz-Cesato cazul \(\left [ \frac{1}{\infty } \right ]\), ipotezele sunt satisfacute, rezultă că 

\begin{displaymath}
  \lim_{n \to \infty }\frac{\ln x_{n}}{n} 
= \lim_{n \to \infty }\frac{\ln x_{n}}{n}  
= \lim_{n \to \infty}\frac{x_{n+1}-\ln x_{n}}{n+1-n} 
= \lim_{n \to \infty}\left ( \ln x_{n+1} - \ln x_{n} \right ) 
= \lim_{n \to \infty }\ln \frac{x_{n+1}}{x_{n}} 
= \ln A
\end{displaymath}


De aici deducem că 

\begin{displaymath}
  \lim_{n \to \infty }\sqrt[n]{x_{n}} = \lim_{n \to \infty } e^{\frac{\ln x_{n}}{n}} = e^{\lim_{n \to \infty } \frac{\ln x_{n}}{n}} = e^{\ln A} = A = \lim_{n \to \infty}\frac{x_{n+1}}{n}.
\end{displaymath}









\bibliographystyle{unsrt}
\setlength{\baselineskip}{\normalbaselineskip}
\setlength{\parskip}{0pt}
\bibliography{refs}
\end{document}